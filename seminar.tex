\documentclass[12pt,a4paper]{report}

\usepackage[croatian]{babel}
\usepackage{amsmath, amsthm, amssymb}
\usepackage{epsfig}
\usepackage{graphicx}
\usepackage{hyperref}
\usepackage[left=3.5cm,top=1.5cm,right=3cm,bottom=4cm]{geometry}
\usepackage{mathtools}
\usepackage{fancyhdr}

\usepackage[T1]{fontenc}
% Fonts:
\usepackage{helvet} % Helvetica; phv
\usepackage{palatino} % Palatino; ppl

\gdef \title{Matematika 3, Seminarski Rad}
\gdef \author{Tin Švagelj}
\gdef \degree{Jednopredmetna Informatika}
\gdef \university{
Fakultet Informatike i Digitalnih Tehnologija,\\
Sveučilište u Rijeci
}
\gdef \semguide{dr.sc. Marija Maksimović}
\gdef \author{Tin Švagelj}
\gdef \date{\today}

\graphicspath{ {./images/} }

%\pagestyle{fancy}
%\fancyhead{}
%\header and footer section
%\renewcommand\headrulewidth{0.1pt}
%\fancyhead[L]{\footnotesize \leftmark}
%\fancyhead[R]{\footnotesize \thepage}
%\renewcommand\headrulewidth{0pt}
%\fancyfoot[R]{\small College of Engineering Trivandrum}
%\renewcommand\footrulewidth{0.1pt}
%\fancyfoot[C]{2020 - 2021}
%\fancyfoot[L]{\small \title}

\begin{document}

\bibliography{ref}

\pagenumbering{roman}


\newenvironment{coverpage}
\thispagestyle{empty}
\begin{titlepage}
	\begin{center}
		\vspace*{20pt}
		{\fontfamily{phv} \Large \bf \title \par}
		\large \em \fontfamily{pzc}{Seminarski Rad}\\ [.15\baselineskip] \par

		\vspace{\stretch{0.3}}

		\normalfont voditeljica kolegija:\\
		\bf {\semguide}\\
		\vspace*{10pt}
		\normalfont student:\\
		\bf {\author},\\
		\fontfamily{ppl} {\bfseries \degree}

		\vspace{\stretch{0.25}}
		
		\footnotesize{\bf \university} \par
		\bf{\date}
	\end{center}		
\end{titlepage}	


% \include{coverpage}

\tableofcontents
\newpage

\setcounter{page}{1}
\pagenumbering{arabic}

\chapter{Uvod}

Potrebno je odrediti ekstreme i nacrtati funkciju:
\begin{equation}
f(x, y) = 2x^3 + 2y^3 - 36xy + 430
\end{equation}

Kako bismo mogli odrediti ekstreme multivarijatne funkcije, potrebno je odrediti domenu funkcije.

Zatim je potrebno odrediti parcijalnu derivaciju prvog reda multivarijatne funkcije,
pri čemu (u ovom slučaju) koristimo pravila deriviranja:

\begin{align}
    (c)' &= 0 \text{ i }\\
    (x^a)' &= ax^{a-1}.
\end{align}

, i pravilo o deriviranju zbroja:

\begin{equation}
    (f + g)' = f' + g'.
\end{equation}

Sa parcijalnom drivacijom prvog reda funkcije $f$ po $x$ i drivacijom prvog reda po $y$ određujemo stacionarne točke, rješavajući sustav jednadžbi:
\begin{equation}
    \begin{cases}
        \frac{\partial f}{\partial x} = 0 \\
        \frac{\partial f}{\partial y} = 0
    \end{cases}
\end{equation}

Kako bi utvrdili jesu li dobivene točke sedlaste, minimumi ili maksimumi, potrebna nam je derivacija drugog reda:

\begin{equation}
    \frac{\partial^2 f}{\partial x^2} = \frac{\partial}{\partial x} (\frac{\partial f}{\partial x}).
\end{equation}

Za računanje Hessove matrice također nam je potrebna derivacija $f$ po $x$ i $y$:

\begin{equation}
    \frac{\partial^2 f}{\partial x \partial y} = \frac{\partial}{\partial y} (\frac{\partial f}{\partial x}).
\end{equation}


Za neku funkciju $f(x_1, x_2, \dots, x_n)$, Hessova matrica je definirana kao\\

$\mathbf{H}(f(\mathbf{x, y})) = \mathbf{J}(\nabla f(\mathbf{x, y}))$,

a gradijent funkcije $f$ ($\nabla f(\mathbf{x, y})$):

\begin{equation}
    \nabla f(p) = \begin{bmatrix}
        \frac{\partial f}{\partial x_1}(p) \\
        \vdots \\
        \frac{\partial f}{\partial x_n}(p)
       \end{bmatrix}
\end{equation}

\begin{equation}
    \mathbf H_f= \begin{bmatrix}
        \dfrac{\partial^2 f}{\partial x_1^2} & \dfrac{\partial^2 f}{\partial x_1\,\partial x_2} & \cdots & \dfrac{\partial^2 f}{\partial x_1\,\partial x_n} \\[2.2ex]
        \dfrac{\partial^2 f}{\partial x_2\,\partial x_1} & \dfrac{\partial^2 f}{\partial x_2^2} & \cdots & \dfrac{\partial^2 f}{\partial x_2\,\partial x_n} \\[2.2ex]
        \vdots & \vdots & \ddots & \vdots \\[2.2ex]
        \dfrac{\partial^2 f}{\partial x_n\,\partial x_1} & \dfrac{\partial^2 f}{\partial x_n\,\partial x_2} & \cdots & \dfrac{\partial^2 f}{\partial x_n^2}
    \end{bmatrix}.
\end{equation}

\chapter{Razrada}

S obzirom da je zadana multivarijatna funkcija polinom, važeća za svaki $x \in \mathbb{R}$ i $y \in \mathbb{R}$.
Dakle domena funkcije je skup svih realnih brojeva:
$$
    D(f) = \mathbb{R} \times \mathbb{R} = \mathbb{R}^2
$$

\section{Parcijalne derivacije prvog reda}

\begin{align*}
    \frac{\partial f}{\partial x} & = \frac{\partial}{\partial x} (2x^3 + 2y^3 - 36xy + 430) \\
    & = \frac{\partial}{\partial x} 2x^3 + \frac{\partial}{\partial x}2y^3 - \frac{\partial}{\partial x}36xy + \frac{\partial}{\partial x}430 \\
    & = 6x^2 + 0 - 36y + 0 \\
    & = 6x^2 - 36y \\
    \\
    \frac{\partial f}{\partial y} & = \frac{\partial}{\partial y} (2x^3 + 2y^3 - 36xy + 430) \\
    & = \frac{\partial}{\partial y} 2x^3 + \frac{\partial}{\partial y}2y^3 - \frac{\partial}{\partial y}36xy + \frac{\partial}{\partial y}430 \\
    & = 0 + 6y^2 - 36x + 0\\
    & = 6y^2 - 36x
\end{align*}

\section{Stacionarne točke}

$$
\begin{cases}
    6x^2 - 36y = 0 \\
    6y^2 - 36x = 0
\end{cases}
$$

Izrazimo $y$ iz prvog izraza,
\begin{align*}
    6x^2 - 36y &= 0 \\
    36y & = 6x^2 \\
    y & = \frac{6x^2}{36} = \frac{x^2}{6} \\
\end{align*}

te ga uvrstimo u drugi:
\begin{align*}
    6(\frac{x^2}{6})^2 - 36x & = 0 \\
    x^4 - 36x & = 0
\end{align*}

, jedno rješenje $x_1 = 0$ dobivamo izlučivanjem $x$a iz izraza:
\begin{align*}
    x(x^3 - 36) &= 0 \\
    x_1 &= 0
\end{align*}

drugi član umnoška nam daje drugo rješenje:
\begin{align*}
    x^3 - 36 &= 0 \\
    x^3 &= 36 \\
    x &= \sqrt[3]{36} \\
    x_2 &= \sqrt[3]{36},
\end{align*}

zbog duplog pojavljivanja $x$a u zadanom polinomu se radi o funkciji simetričnoj s obzirom na ishodište te trebamo tretirati drugo rješenje kao da je njegov predznak uklonjen.

Tom logikom dobivamo 3 rješenja sustava:
\begin{center}
\begin{tabular}{c c c}
    $x_1 = 0$ & $x_2 = \sqrt[3]{36}$ & $x_3 = -\sqrt[3]{36}$\\
\end{tabular}
\end{center}

\begin{align*}
    6x^2 - 36y &= 0\\
    6(\sqrt[3]{36})^2 - 36y &= 0\\
    36y &= 6(\sqrt[3]{36^2})\\
    6y &= \sqrt[3]{36^2} \\
    y &= \frac{\sqrt[3]{6^4}}{6} \\
    y &= 6^{\frac{4}{3}} * 6^{-1} = 6^{\frac{4}{3}} * 6^{-\frac{3}{3}} = 6^{\frac{4}{3} - \frac{3}{3}} \\
    y &= \sqrt[3]6 \\
\end{align*}

\begin{center}
\begin{tabular}{c c}
    $x$ & $y$ \\
    $0$ & $0$ \\
    $\sqrt[3]{36}$ & $\sqrt[3]6$ \\
    $-\sqrt[3]{36}$ & $-\sqrt[3]6$ \\
\end{tabular}
\end{center}

\section{Parcijalne derivacije drugog reda}

\begin{align*}
    \frac{\partial^2 f}{\partial x^2} & = \frac{\partial}{\partial x} (6x^2 - 36y)\\
    & = \frac{\partial}{\partial x} 6x^2 - \frac{\partial}{\partial x} 36y\\
    & = 12x - 0\\
    & = 12x\\
    \\
    \frac{\partial^2 f}{\partial y^2} & = \frac{\partial}{\partial x} (6y^2 - 36x)\\
    & = \frac{\partial}{\partial y} 6y^2 - \frac{\partial}{\partial y} 36x\\
    & = 12y - 0\\
    & = 12y\\
    \\
    \frac{\partial^2 f}{\partial x \partial y} & = \frac{\partial}{\partial y} (6x^2 - 36y) \\
    & = \frac{\partial}{\partial y} 6x^2 - \frac{\partial}{\partial y} 36y \\
    & = 0 - 36 \\
    & = -36
\end{align*}

Određujemo vrijednost druge parcijalne derivacije svake stacionarne točke kako bi mogli odrediti Hessovu matricu:

$$H_f = \begin{bmatrix} 12x & -36 \ -36 & 12y \end{bmatrix}$$

Određujemo svojstvene vrijednosti Hessove matrice za svaku stacionarnu točku:

%\begin{itemize}
%    \item Za (0, 0), svojstvene vrijednosti su 0 i 12; dobivena stacionarna točka sedlasta.
%    \item Za (3, 3), svojstvene vrijednosti su 12 i 12; stacionarna točka je lokalni maksimum.
%    \item Za (-3, -3), svojstvene vrijednosti su 12 i 12; stacionarna točka je lokalni minimum.
%\end{itemize}

%Dakle, na osnovu prethodnog računa zaključujemo da su ekstremi funkcije $2x^3 + 2y^3 - 36xy + 430$ točke
%\begin{itemize}
%    \item (3, 3) kao lokalni maksimum
%    \item i (-3, -3) kao lokalni minimum.
%\end{itemize}

Korištenjem \verb|numpy| i \verb|matplotlib| biblioteka u Pythonu, pomoću sljedećeg koda možemo dobiti 3D prikaz cijele multivarijatne funkcije:

\begin{verbatim}
    import numpy as np
    import matplotlib.pyplot as plt
    from mpl_toolkits.mplot3d import Axes3D
    
    def f(x, y):
        return 2*x**3 + 2*y**3 - 36*x*y + 430
    
    x = np.linspace(-10, 10, 100)
    y = np.linspace(-10, 10, 100)
    X, Y = np.meshgrid(x, y)
    Z = f(X, Y)
    
    fig = plt.figure()
    ax = fig.add_subplot(111, projection='3d')
    ax.plot_surface(X, Y, Z)
    
    plt.show()
\end{verbatim}

\noindent\makebox[\textwidth]{\includegraphics[width=300px]{graf}}
\newpage

\chapter{Zaključak}

\newpage

\pagenumbering{roman}
\setcounter{page}{2}

\def\bibname{Literatura}
\printbibliography
\end{document}
