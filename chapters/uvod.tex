\chapter{Uvod}

Potrebno je odrediti ekstreme i nacrtati funkciju:
\begin{equation}
f(x, y) = 2x^3 + 2y^3 - 36xy + 430
\end{equation}

Bitno je odrediti {\bf domenu funkcije} na početku kako bismo uspješno i smisleno odredili ekstreme funkcije jer mogu postojati samo unutar njene domene.

Kako bismo mogli odrediti ekstreme funkcije, potrebno je odrediti {\bf nulte točke gradijenta} ($\nabla$) funkcije $f$ \cite{ccalc}:
\begin{equation}
    \nabla f(\mathbf{x}, \mathbf{y}) = \begin{bmatrix}
        \frac{\partial f}{\partial x} \\
        \frac{\partial f}{\partial y}
    \end{bmatrix}\textnormal{,}
\end{equation}

rješavajući sustav jednadžbi:
\begin{equation}
    \begin{cases}
        \frac{\partial f}{\partial x} = 0 \\
        \frac{\partial f}{\partial y} = 0 \textnormal{,}
    \end{cases}
\end{equation}

gdje su rješenje nulte točke funkcije gradijenta ($\nabla f(\mathbf{x}, \mathbf{y}) = 0$), tj. ekstremi funkcije $f$.

Za određivanje parcijalne derivacije prvog reda funkcije koristimo {\bf pravila deriviranja} \cite{kolegij}:

\begin{align}
    (c)' &= 0 \text{ i }\\
    (x^a)' &= ax^{a-1}\\
    (f + g)' &= f' + g' \textnormal{.}
\end{align}

\vspace*{20pt}

Crtanje će biti izvedeno koristeći \verb|NumPy| i \verb|Matplotlib| biblioteke u Python programskom jeziku.\par

\newpage
