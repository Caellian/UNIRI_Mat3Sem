\chapter{Razrada}

S obzirom da je zadana funkcija $f$ (1.1) racionalna, važeća je za svaki $x \in \mathbb{R}$ i $y \in \mathbb{R}$ \cite[vidi][stranica 119]{kolegij}.
Dakle domena funkcije je skup svih realnih brojeva:
$$
    D(f) = \mathbb{R} \times \mathbb{R} = \mathbb{R}^2
$$

\section{Parcijalne derivacije prvog reda}

Kako bismo odredili gradijent funkcije $f$, trebamo prvo odrediti parcijalne derivacije te funkcije po $x$ i $y$, pri čemu se služimo pravilima 1.4, 1.5 i 1.6:

\begin{align*}
    \frac{\partial f}{\partial x} & = \frac{\partial}{\partial x} (2x^3 + 2y^3 - 36xy + 430) \\
    & = \frac{\partial}{\partial x} 2x^3 + \frac{\partial}{\partial x}2y^3 - \frac{\partial}{\partial x}36xy + \frac{\partial}{\partial x}430 \\
    & = 6x^2 + 0 - 36y + 0 \\
    & = 6x^2 - 36y \\
    \\
    \frac{\partial f}{\partial y} & = \frac{\partial}{\partial y} (2x^3 + 2y^3 - 36xy + 430) \\
    & = \frac{\partial}{\partial y} 2x^3 + \frac{\partial}{\partial y}2y^3 - \frac{\partial}{\partial y}36xy + \frac{\partial}{\partial y}430 \\
    & = 0 + 6y^2 - 36x + 0\\
    & = 6y^2 - 36x \textnormal{.}
\end{align*}

S obzirom da je domena zadane funkcije $f$ skup svih realnih brojeva, ne trebamo isključiti rješenja iz dobivenih izraza.

\section{Gradijent funkcije}

Određujemo gradijent funkcije $f$ uz dobivene parcijalne derivacije na osnovu formule 1.2:

\begin{equation}
    \nabla f(\mathbf{x}, \mathbf{y}) = \begin{bmatrix}
        6x^2 - 36y \\
        6y^2 - 36x
    \end{bmatrix}\textnormal{,}
\end{equation}

\section{Ekstremi}

Gradijent 2.1 izjednačujemo s nulom kako bi mu odredili nulte točke tj. ekstreme funkcije.
To možemo izraziti kao sustav (1.3):

$$
\begin{cases}
    6x^2 - 36y = 0 \\
    6y^2 - 36x = 0
\end{cases}
$$

Izrazimo $y$ iz prvog izraza,
\begin{align*}
    6x^2 - 36y &= 0 \\
    36y & = 6x^2 \\
    y & = \frac{6x^2}{36} = \frac{x^2}{6} \\
\end{align*}

te ga uvrštavamo u drugi kako bismo dobili jednadžbu za $x$eve nultih točaka gradijenta:
\begin{align}
    6(\frac{x^2}{6})^2 - 36x & = 0 \nonumber \\
    x^4 - 36x & = 0 \textnormal{.}
\end{align}

Jedno rješenje ($x_1$) uočavamo nakon izlučivanja $x$a iz izraza:
\begin{align*}
    x(x^3 - 36) &= 0 \\
    x_1 &= 0 \textnormal{.}
\end{align*}

Zaključujemo da je jedno rješenje $x_1 = 0$ jer ako je $x$ jednak nuli, onda će cijeli izraz 2.2 biti jednak nuli. \par

Drugi član umnoška nam daje drugo rješenje:
\begin{align*}
    x^3 - 36 &= 0 \\
    x^3 &= 36 \\
    x &= \sqrt[3]{36} \textnormal{,}
\end{align*}

ali zbog duplog pojavljivanja $x$a u zadanom polinomu se radi o funkciji simetričnoj s obzirom na ishodište te je zbog toga dobiveno rješenje zapravo dvojno rješenje:

$$
x_{2,3} = \pm \sqrt[3]{36} \textnormal{.}
$$

Time dobivamo 3 rješenja sustava za $x$:
\begin{center}
\begin{tabular}{c c c}
    $x_1 = 0$ & $x_2 = \sqrt[3]{36}$ & $x_3 = -\sqrt[3]{36}$\\
\end{tabular}
\end{center}

Za $x_1 = 0$, $y$ će biti 0. $y_1$ i $y_2$ računamo izjednačavajući dobivene vrijednosti sa jednom od parcijalnih derivacija:

\begin{align*}
    6x^2 - 36y &= 0\\
    6(\sqrt[3]{36})^2 - 36y &= 0\\
    36y &= 6(\sqrt[3]{36^2})\\
    6y &= \sqrt[3]{36^2} \\
    y &= \frac{\sqrt[3]{6^4}}{6} \\
    y &= 6^{\frac{4}{3}} * 6^{-1} = 6^{\frac{4}{3}} * 6^{-\frac{3}{3}} = 6^{\frac{4}{3} - \frac{3}{3}} \\
    y &= \sqrt[3]6
\end{align*}

Time utvrđujemo da su ekstremi funkcije $f$:

\begin{center}
\begin{tabular}{c c}
    $x$ & $y$ \\
    $0$ & $0$ \\
    $\sqrt[3]{36}$ & $\sqrt[3]6$ \\
    $-\sqrt[3]{36}$ & $-\sqrt[3]6$ \\
\end{tabular}
\end{center}

\newpage